\newglossaryentry{repl}{
name=Read Evaluate Print Loop,
description={Pattern zum Erzeugen einer Konsole, die in einer Endlosschleife Eingaben liest, die auswertet und das Ergebnis wieder ausgibt}
}

\newglossaryentry{sla}{
name=Service Level Agreement,
description={Übereinkunft zwischen dem Anbieter und dem Nutzer eines Dienstes über dessen Qualität (z.B. Antwortzeiten, Durchsatz, Verfügbarkeit, etc.)}
}

\newglossaryentry{worker}{
name=Worker,
description={Host, der als Arbeitsknoten in einem Rechnercluster dient. Falls nicht anders beschrieben ist hier ein Rechner gemeint, der seine Ressourcen einer Spark-Anwendung zur Verfügung stellt und mit seinem Festspeicher Teil eines verteilten Dateisystems ist}
}

\newglossaryentry{master}{
name=Master,
description={Host, der Verwaltungsaufgaben innerhalb eines Rechnerclusters übernimmt und dazu mit hierarchisch untergeordneten Rechnern kommuniziert. Zu den Aufgaben kann insbesondere das Verteilen von Arbeitsaufträgen oder Speicherblocks gehören}
}

\newglossaryentry{cluster}{
name=Rechnercluster,
description={Vernetzter Verbund aus eigenständig lauffähigen Rechnern}
}

\newglossaryentry{rj45}{
name=RJ45,
description={Achtpolige Modularsteckverbindung zur Datenübertragung}
}

\newglossaryentry{tweet}{
name=Tweet,
description={Kurznachricht auf der Plattform des Anbieters Twitter (\cite{twi}). In dieser Nachricht stehen maximal 140 Zeichen Text. Diese Nachrichten können von jedem Benutzerkonto der Plattform erstellt werden}
}

\newglossaryentry{dataframe}{
name={DataFrame},
description={Abstrakte Datenstruktur von Spark, die den Zugriff auf strukturierte Daten über einen Sparkcluster ermöglicht.}
}


\newacronym{rdd}{RDD}{Resilient Distributed Dataset}

\newacronym{dsm}{DSM}{Distributed Shared Memory}

\newacronym{api}{API}{Application Programming Interface}
