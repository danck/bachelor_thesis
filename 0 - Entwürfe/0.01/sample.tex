\documentclass[draft=false
              ,paper=a4
              ,twoside=false
              ,fontsize=11pt
              ,headsepline
              ,BCOR10mm
              ,DIV11
              ]{scrbook}
\usepackage{graphicx}
\usepackage[ngerman,english]{babel}
%% see http://www.tex.ac.uk/cgi-bin/texfaq2html?label=uselmfonts
\usepackage[T1]{fontenc}
%\usepackage[utf8]{inputenc}
\usepackage[latin1]{inputenc}
\usepackage{libertine}
\usepackage{pifont}
\usepackage{microtype}
\usepackage{textcomp}
\usepackage[german,refpage]{nomencl}
\usepackage{setspace}
\usepackage{makeidx}
\usepackage{listings}
\usepackage{natbib}
\usepackage[ngerman,colorlinks=true]{hyperref}
\usepackage{soul}
\usepackage{hawstyle}
\usepackage{lipsum} %% for sample text

\usepackage{svg}
\usepackage{amsmath}
\usepackage{xcolor}

\setsvg{inkscape={"C:/Program Files/Inkscape/inkscape.exe"= -z -C}}

%% define some colors
\colorlet{BackgroundColor}{gray!20}
\colorlet{KeywordColor}{blue}
\colorlet{CommentColor}{black!60}
%% for tables
\colorlet{HeadColor}{gray!60}
\colorlet{Color1}{blue!10}
\colorlet{Color2}{white}

%% configure colors
\HAWifprinter{
  \colorlet{BackgroundColor}{gray!20}
  \colorlet{KeywordColor}{black}
  \colorlet{CommentColor}{gray}
  % for tables
  \colorlet{HeadColor}{gray!60}
  \colorlet{Color1}{gray!40}
  \colorlet{Color2}{white}
}{}
\lstset{%
  numbers=left,
  numberstyle=\tiny,
  stepnumber=1,
  numbersep=5pt,
  basicstyle=\ttfamily\small,
  keywordstyle=\color{KeywordColor}\bfseries,
  identifierstyle=\color{black},
  commentstyle=\color{CommentColor},
  backgroundcolor=\color{BackgroundColor},
  captionpos=b,
  fontadjust=true
}
\lstset{escapeinside={(*@}{@*)}, % used to enter latex code inside listings
        morekeywords={uint32_t, int32_t}
}
\ifpdfoutput{
  \hypersetup{bookmarksopen=false,bookmarksnumbered,linktocpage}
}{}

%% more fancy C++
\DeclareRobustCommand{\cxx}{C\raisebox{0.25ex}{{\scriptsize +\kern-0.25ex +}}}

\clubpenalty=10000
\widowpenalty=10000
\displaywidowpenalty=10000

% unknown hyphenations
\hyphenation{
}

%% recalculate text area
\typearea[current]{last}

\makeindex
\makenomenclature

\begin{document}
\selectlanguage{ngerman}

%%%%%
%% customize (see readme.pdf for supported values)
\HAWThesisProperties{Author={Daniel Kirchner}
                    ,Title={Skalierbare Datenanalyse mit Apache Spark}
										,SubTitle={Evaluation von Anwendungsf�llen aus Klimaforschung und Text-Mining}
                    ,EnglishTitle={Scalable Data Analysis with Apache Spark}
                    ,ThesisType={Bachelorarbeit}
                    ,ExaminationType={Bachelorpr�fung}
                    ,DegreeProgramme={Bachelor of Science Angewandte Informatik}
                    ,ThesisExperts={Prof. Dr. Kahlbrandt \and Prof. Dr. Zweitpr�fer}
                    ,ReleaseDate={1. Januar 2345}
                  }

%% title
\frontmatter

%% output title page
\maketitle

\onehalfspacing

%% add abstract pages
%% note: this is one command on multiple lines
\HAWAbstractPage
%% German abstract
{Schl�sselwort 1, Schl�sselwort 2}%
{Dieses Dokument \ldots}
%% English abstract
{keyword 1, keyword 2}%
{This document \ldots}

\newpage
\singlespacing

\tableofcontents

\newpage
%% enable if these lists should be shown on their own page
%%\listoftables
%%\listoffigures
\lstlistoflistings

%% main
\mainmatter
\onehalfspacing
%% write to the log/stdout
\typeout{===== File: chapter 1}
%% include chapter file (chapter1.tex)
%%\include{chapter1}

%%%%
%% add some text to generate a sample document
\chapter{Einf\"uhrung}

\section{Motivation}

Die Entwicklung und Verbesserung von Frameworks zur Verarbeitung gro�er Datenmengen ist zur Zeit hochaktuell und sehr im Fokus von Medien und Unternehmen [VERWEIS]. Verschiedene Programme und Paradigmen konkurrieren um die schnellste, bequemste und stabilste Art gro�en Datenmengen einen gesch�ftsf�rdenden Nutzen abzuringen.
\break

Unter dem Begriff "`gro�e Datenmengen"' oder "`Big Data"' werden solche Datenmengen zusammengefasst, die die Kriterien Volume, Velocity, Variety [VERWEIS, Doug Laney] erf�llen oder "`Datenmengen, die nicht mehr unter Auflage bestimmter SLAs auf einzelnen Maschinen verarbeitet werden k�nnen"' [VERWEIS, Hadoop/Yarn Entwickler].

Als Unternehmen, das fr�h mit solchen Datenmengen konfrontiert war implementierte Google das Map-Reduce Paradigma [VERWEIS] als Framework zur Ausnutzung vieler kosteng�nstiger Rechner um Webseiten einzustufen und f�r andere Aufgaben [VERWEIS]. 

In Folge der Ver�ffentlichung ihrer Idee im Jahr 2005 [VERWEIS] wurde Map-Reduce in Form der OpenSource Implementation Hadoop (gemeinsam mit einer Implementation des Google File Systems GFS, u.a.) [VERWEIS] zum de-facto Standard f�r Big-Data-Analyseaufgaben [VERWEIS?].
\break

Reines Map-Reduce (nach Art von Hadoop) als Programmierparadigma zur Verarbeitung gro�er Datenmengen zeigt jedoch in vielen Anwendungsf�llen Schw�chen:
\begin{itemize}
	\item Daten, die in hoher Frequenz entstehen und schnell verarbeitet werden sollen erfordern h�ufiges Neustarten von Map-Reduce-Jobs.
	\item Algorithmen die w�hrend ihrer Ausf�hrung iterativ Zwischenergebnisse erzeugen und auf vorherige angewiesen sind (typischerweise Maschinenlernalgorithmen) k�nnen nur durch persistentes Speichern der Daten und wiederholtes Auslesen zwischen allen Iterationsschritten implementiert werden.
	\item Generell erfolgen Anfragen an ein solches System immer in Form von kleinen Programmen. Dieses Verfahren ist offensichtlich nicht so deklarativ und leicht erlernbar wie beispielsweise SQL-Anfragen an klassische Datenbanken.
\end{itemize}

In der Folge entstanden viele Ans�tze dieses Paradigma zu ersetzen, zu erg�nzen oder durch �bergeordnete Ebenen und High-Level-APIs zu vereinfachen.

\begin{itemize}
	\item {[}VERWEIS: A survey of large scale...{]} oder Aufz�hlung.
\end{itemize}

Eine der Alternativen zu der Map-Reduce-Komponente in Hadoop die "`general engine for large-scale data processing"' Apache Spark.

Ein Indiz f�r das steigende Interesse an diesem Produkt liefert unter anderem ein Vergleich des Interesses an Hadoop und Spark auf Google:
\break

\includegraphics[scale=0.4]{bilder/trends_spark_vs_hadoop.PNG}

\section{Kontextabgrenzung}
Das Ziel dieser Arbeit ist es einen Einblick in die grundlegenden Konzepte und Anwendungsm�glichkeiten von Apache Spark zu vermitteln.

F�r ein tieferes Verst�ndnis werden zwei Anwendungsf�lle untersucht und deren L�sung detailiert dokumentiert und bewertet.
\break

Nur am Rande wird betrachtet:
\begin{itemize}
	\item Vergleich mit �hnlichen Produkten
	\item Empirische Messung des Skalierungsverhaltens
	\item Konkrete Hinweise zu Installation und Nutzung
\end{itemize}

\chapter{Vorstellung von Apache Spark}

\section{�berblick}

\section{Kernkonzepte}
\textcolor{gray}{--- Warum ist Spark so schnell (und wo vielleicht nicht)? ---}
\subsection{Resilient Distributed Datasets}
\subsection{Lineage}
\subsection{DAG Scheduler}

\section{Standardbibliotheken}
\textcolor{gray}{--- Warum ist Spark so einfach (und wo vielleicht nicht)? ---}
\subsection{Spark SQL}
\subsection{MLlib}
\subsection{Streaming}
\subsection{GraphX}

\section{Entwicklergemeinschaft}

\section{Verwandte Produkte}
\textcolor{gray}{--- Erg�nzende oder konkurrierende Produkte ---}
\subsection{YARN}
\subsection{Mesos}
\subsection{Flink}


\chapter{Untersuchung von Anwendungsf�llen}
Im Folgenden wird Apache Spark im Rahmen zweier grunds�tzlich verschiedener Anwendungsf�lle betrachtet.
\break

Beispiel 1: Eine typische Anwendung mit verteilten lokalem Storage (HDFS) und Spark als "`Client"' eines bestehenden Yarn Clustermanagers. \textcolor{gray}{--- Commodity Hardware (Rasperry Pi Cluster). ---}
\break

Beispiel 2: Eine untypische Anwendung mit verteiltem entfernten Storage und dem Spark Standalone Clustermanager. \textcolor{gray}{--- HPC Hardware ("`Thunder"' des Hamburger KlimaCampus). ---}

\section{Identifikation von Hot Topics in der Spark Community}
\textcolor{gray}{--- Fusion von Tweets und Mailinglisten 
https://spark.apache.org/docs/1.3.0/mllib-feature-extraction.html
Implementation auf einem Raspberry Pi Cluster mit HDFS und Yarn Clustermanager ---}
\subsection{Beschreibung des Problems}

\subsection{Hardwarekontext und Performance-Basisdaten}
\textcolor{gray}{--- hier kommen die eingesetzten systeme, und relevante laufzeitmessungen (netzwerk, storage, cpu) hin ---}

%%\begin{figure}[htbp]
%%  \centering
  \includesvg[width=\paperwidth]{versuchsaufbau}
%%  \caption{svg image}
%%\end{figure}

\subsection{Architektur�bersicht}
\textcolor{gray}{--- hier kommen Verteilungs- und Komponentendiagramm hin ---}
\subsection{Detailierte L�sungsbeschreibung}
\textcolor{gray}{--- hier kommen laufzeitdiagramme und codeschnipsel hin ---}
\subsection{Ergebnisse}
\textcolor{gray}{--- Tabellen und Diagramme Ergebnissen, evt. Skalierungsverhalten ---}
\textcolor{gray}{--- Bewertung ---}

\section{Evaluierung einer spark-basierten Implementation von CDOs auf einem HPC Cluster mit nicht-lokalem Storage}
\textcolor{gray}{--- Implementation ausgew�hlter CDOs (sehr wenige, m�glicherweise nur 1-2) mit der Core-API von Spark. Testlauf auf einem HPC Cluster mit nicht-lokalem, allerdings per Infiniband angeschlossenen Storage.
Insbesondere Betrachtung des Skalierungsverhaltens und der "`Sinnhaftigkeit"'. ---}

\subsection{Beschreibung des Problems}
\textcolor{gray}{--- Erl�uterung von CDOs (Climate Data Operators). ---}
\subsection{Hardwarekontext und Performance-Basisdaten}
\textcolor{gray}{--- hier kommen die eingesetzten systeme, und relevante laufzeitmessungen (netzwerk, storage, cpu) hin ---}
\subsection{Architektur�bersicht}
\textcolor{gray}{--- hier kommen Verteilungs- und Komponentendiagramm hin ---}
\subsection{Detailierte L�sungsbeschreibung}
\textcolor{gray}{--- hier kommen laufzeitdiagramme und codeschnipsel hin ---}
\subsection{Ergebnisse}
\textcolor{gray}{--- Tabellen und Diagramme Ergebnissen, evt. Skalierungsverhalten ---}
\textcolor{gray}{--- Bewertung ---}

\chapter{Schlussbetrachtung}
\section{Kritische W\"urdigung der Ergebnisse}
\section{Ausblick und offene Punkte}

%%\lipsum

See also \cite{sample_bib}.
%%%%

%% appendix if used
%%\appendix
%%\typeout{===== File: appendix}
%%\include{appendix}

% bibliography and other stuff
\backmatter

\typeout{===== Section: literature}
%% read the documentation for customizing the style
\bibliographystyle{dinat}
\bibliography{sample}

\typeout{===== Section: nomenclature}
%% uncomment if a TOC entry is needed
%%\addcontentsline{toc}{chapter}{Glossar}
\renewcommand{\nomname}{Glossar}
\clearpage
\markboth{\nomname}{\nomname} %% see nomencl doc, page 9, section 4.1
\printnomenclature

%% index
\typeout{===== Section: index}
\printindex

\HAWasurency

\end{document}
