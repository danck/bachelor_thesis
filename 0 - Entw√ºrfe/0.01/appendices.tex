\begin{appendices}
\section{Installation der Plattform}
\section{Quellcode (Auszüge)}
\subsection{Performance-Messungen}

\begin{lstlisting}[language=Bash,caption={Messung der Festplattenperformance - Beispiel: Schreiben eines 512MB Blockes},label={lst:measure_harddrive}]
echo 3 | sudo tee /proc/sys/vm/drop_caches
dd if=/dev/zero of=test512.out bs=512MB count=1
\end{lstlisting}

\subsection{Realisierung einer einfachen Continuous Deployment Pipeline}\label{subsec:pipeline}
\textit{post-receive}-Hook von dem Git-Repository\footnote{https://git-scm.com/, abgerufen am 06.06.2015} der ModelBuilder-Komponente
\begin{lstlisting}[language=Bash,caption={Primitive Continuous Deployment Pipeline. Beispiel: ModelBuilder},label={lst:cdp_modelbuilder}]
#/bin/bash

export SPARK_HOME=/opt/spark/

# clean up previous build
rm -rf ~/autobuilds/model_builder
cd ~/autobuilds
git clone ~/git/model_builder
cd model_builder

# run build
sbt package
if [ $? -ne "0" ]; then exit 1; fi

# run test suite
sbt test
if [ $? -ne "0" ]; then exit 1; fi

# deploy to cluster
/opt/spark/bin/spark-submit --class "de.haw.bachelorthesis.dkirchner\
.ModelBuilder" --master spark://192.168.206.131:7077 --driver-memory\
256m --executor-memory 384m \
~/autobuilds/model_builder/ [...] /model-builder_2.10-1.0.jar\
hdfs://192.168.206.131:54310/user/daniel/user_emails_corpus1.txt\
<emailaccount> <passwort>
\end{lstlisting}

\section{Sonstiges}
\subsection{Einschätzung des theoretischen Spitzendurchsatzes von Mittelklasse-Servern}
\label{subsec:commodity_servers}
Um zu einer groben Einschätzung des möglichen Datendurchsatzes verschiedener Schnittstellen bei "`Commodity Servern"' zu gelangen, wurden drei Systeme von großen Herstellern ausgewählt.\\
In der Grundkonfiguration kosten diese Systeme (zum Zeitpunkt dieser Arbeit) um die € 2000,- und lassen damit auf die Größenordnungen bei dem Datendurchsatz bestimmter Schnittstellen bei preisgünstigen Mehrzweck-Rechenknoten schließen.

\begin{table}[ht]
	\centering % used for centering table
	\begin{tabular}{c c c c} % centered columns (4 columns)
		\hline\hline %inserts double horizontal lines
		Modell & Netzwerkschnittstelle & Festspeicher & Arbeitsspeicher\\ [0.5ex] % inserts table
		%heading
		\hline % inserts single horizontal line
		Dell PowerEdge R530 & 1Gb/s Ethernet & PCIe 3.0 & DDR4\\ 
		HP Proliant DL160 Gen8 & 1Gb/s Ethernet & PCIe 3.0 & DDR3\\ 
		System x3650 M5 & 1Gb/s Ethernet & PCIe 3.0 & DDR4\\ % inserting body of the table
		\hline %inserts single line
	\end{tabular}
	\caption{Theoretische Spitzenleistungen bei Mehrzweck-Servern der 2000 Euro Klasse} % title of Table
	\label{table:vglinterfaces} % is used to refer this table in the text
\end{table}

Mit \cite{PCI14} und \cite{Fuj11} lassen sich grobe obere Abschätzungen errechnen, die in Tabelle~\ref{table:vgldurchsatz} angegeben sind.

\end{appendices}