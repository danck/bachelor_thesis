\documentclass[draft=false
              ,paper=a4
              ,twoside=false
              ,fontsize=11pt
              ,headsepline
              ,BCOR10mm
              ,DIV11
              ]{scrbook}
\usepackage[ngerman,english]{babel}
%% see http://www.tex.ac.uk/cgi-bin/texfaq2html?label=uselmfonts
\usepackage[T1]{fontenc}
%\usepackage[utf8]{inputenc}
\usepackage[latin1]{inputenc}
\usepackage{libertine}
\usepackage{pifont}
\usepackage{microtype}
\usepackage{textcomp}
\usepackage[german,refpage]{nomencl}
\usepackage{setspace}
\usepackage{makeidx}
\usepackage{listings}
\usepackage{natbib}
\usepackage[ngerman,colorlinks=true]{hyperref}
\usepackage{soul}
\usepackage{hawstyle}
\usepackage{lipsum} %% for sample text

%% define some colors
\colorlet{BackgroundColor}{gray!20}
\colorlet{KeywordColor}{blue}
\colorlet{CommentColor}{black!60}
%% for tables
\colorlet{HeadColor}{gray!60}
\colorlet{Color1}{blue!10}
\colorlet{Color2}{white}

%% configure colors
\HAWifprinter{
  \colorlet{BackgroundColor}{gray!20}
  \colorlet{KeywordColor}{black}
  \colorlet{CommentColor}{gray}
  % for tables
  \colorlet{HeadColor}{gray!60}
  \colorlet{Color1}{gray!40}
  \colorlet{Color2}{white}
}{}
\lstset{%
  numbers=left,
  numberstyle=\tiny,
  stepnumber=1,
  numbersep=5pt,
  basicstyle=\ttfamily\small,
  keywordstyle=\color{KeywordColor}\bfseries,
  identifierstyle=\color{black},
  commentstyle=\color{CommentColor},
  backgroundcolor=\color{BackgroundColor},
  captionpos=b,
  fontadjust=true
}
\lstset{escapeinside={(*@}{@*)}, % used to enter latex code inside listings
        morekeywords={uint32_t, int32_t}
}
\ifpdfoutput{
  \hypersetup{bookmarksopen=false,bookmarksnumbered,linktocpage}
}{}

%% more fancy C++
\DeclareRobustCommand{\cxx}{C\raisebox{0.25ex}{{\scriptsize +\kern-0.25ex +}}}

\clubpenalty=10000
\widowpenalty=10000
\displaywidowpenalty=10000

% unknown hyphenations
\hyphenation{
}

%% recalculate text area
\typearea[current]{last}

\makeindex
\makenomenclature

\begin{document}
\selectlanguage{ngerman}

%%%%%
%% customize (see readme.pdf for supported values)
\HAWThesisProperties{Author={Daniel Kirchner}
                    ,Title={Skalierbare Datenanalyse mit Apache Spark}
										,SubTitle={Beispielimplementation eines Influenza-Fr�hwarnsystems}
                    ,EnglishTitle={Scalable Data Analysis with Apache Spark}
                    ,ThesisType={Bachelorarbeit}
                    ,ExaminationType={Bachelorpr�fung}
                    ,DegreeProgramme={Bachelor of Science Angewandte Informatik}
                    ,ThesisExperts={Prof. Dr. Kahlbrandt \and Prof. Dr. Zweitpr�fer}
                    ,ReleaseDate={1. Januar 2345}
                  }

%% title
\frontmatter

%% output title page
\maketitle

\onehalfspacing

%% add abstract pages
%% note: this is one command on multiple lines
\HAWAbstractPage
%% German abstract
{Schl�sselwort 1, Schl�sselwort 2}%
{Dieses Dokument \ldots}
%% English abstract
{keyword 1, keyword 2}%
{This document \ldots}

\newpage
\singlespacing

\tableofcontents

\newpage
%% enable if these lists should be shown on their own page
%%\listoftables
%%\listoffigures
\lstlistoflistings

%% main
\mainmatter
\onehalfspacing
%% write to the log/stdout
\typeout{===== File: chapter 1}
%% include chapter file (chapter1.tex)
%%\include{chapter1}

%%%%
%% add some text to generate a sample document
\chapter{Einf\"uhrung}

\section{Motivation}
Der Bedarf auf gro�en Datenmengen zu operieren ist nicht neu. Sp�testens seit in den sp\"aten Neunzigerjahren Suchmaschinenanbieter mit Mengen von Daten und Anfragen konfrontiert wurden, die eine nicht mehr wirtschaftlich durch einzelne Rechner zu bew\"altigen waren, wurden neue Verfahren ben�tigt. 
Algorithmen wurden nun auf die Eigenschaft optimiert m�glichst effizient und fehlertolerant auf verschiedene Maschinen verteilbar zu sein.
\break

Inzwischen ist die Analyse gro�er Datenmengen auch f�r Unternehmen und Einrichtungen interessant geworden, deren Kerngesch\"aft nicht die Daten selbst sind. Regierungen, Wissenschaftler, Industrieunternehmen, Milit�rs, Handelssoftware und viele andere treffen Entscheidungen auf Grundlage von Daten die die Kapazit�ten einzelner Systeme weit �beschreiten.
\break

St\"andige Ver\"anderung und Unvorhersehbarkeit der Anforderungen sind allt\"agliche Praxis. Daten denen man in dem Moment keine Bedeutung beimisst, k�nnen sich in Zukunft als sehr kritisch erweisen und in einem anderen Kontext eine wichtige Rolle spielen. 
Dabei hat sich ein Paradigma als besonders wirksam herausgestellt: Daten werden erst durch die Abfrage in eine h�here Struktur gebracht, w�hrend bei der Speicherung nur ein Minimum an Struktur eingefordert wird.
\break

Ein weiteres Problem ist die Vielfalt der m�glichen Abfragen. F�r bestimmte Aufgaben gen\"ugen Anfragen wie an eine klassische Datenbank. F\"ur andere Probleme sind m�glicherweise komplexere Graph-Analysen, das Anlernen von Maschinenlernalgorithmen oder eine Quasi-Echtzeit-Auswertung von Datenstr�men gefordert.
An dieser Stelle kommt Apache Spark ins Spiel, dass einen Versuch macht alle bisher genannten Probleme zu l�sen.


\section{Kontextabgrenzung}

\section{Relevante Produkte und Meilensteine}

\subsection{\"Uberblick}
\subsection{Big Table}
\subsection{Map/Reduce}
\subsection{Hadoop}

\chapter{Vorstellung von Apache Spark}

\section{\"Ubersicht}
\subsection{Architektur�bersicht}
\subsection{Standardbibliotheken}
\subsubsection{Spark SQL}
\subsubsection{MLlib}
\subsubsection{Streaming}
\subsubsection{GraphX}

\section{Wesentliche Konzepte}
\subsection{Abgrenzung zu Hadoop}
\subsection{Resilient Distributed Datasets}


\chapter{Vorstellung des Beispiels}
\section{Aufgabenbeschreibung}
\section{L\"osungsidee}
\subsection{1. Schritt: \"Ahnlichkeitsma{\ss} f�r W\"orter erzeugen}
\subsection{2. Schritt: Echtzeitbewertung von Textnachrichten aus einem Datenstrom}


\chapter{Implementation und Bewertung}
\section{Technischer Rahmen}
\subsection{OpenStack}
\section{Architektur�bersicht}
\section{Architekturdetails}
\subsection{Modell f\"ur \"Ahnlichkeit von W\"ortern mit MLlib erzeugen}
\subsection{Einlesen von Nachrichten aus dem Twitter Livestream}
\subsection{Verarbeiten und Bewerten der Nachrichten}
\section{Bewertung der Verfahren}

\chapter{Schlussbetrachtung}
\section{Kritische W\"urdigung der Ergebnisse}
\section{Ausblick und offene Punkte}

%%\lipsum

See also \cite{sample_bib}.
%%%%

%% appendix if used
%%\appendix
%%\typeout{===== File: appendix}
%%\include{appendix}

% bibliography and other stuff
\backmatter

\typeout{===== Section: literature}
%% read the documentation for customizing the style
\bibliographystyle{dinat}
\bibliography{sample}

\typeout{===== Section: nomenclature}
%% uncomment if a TOC entry is needed
%%\addcontentsline{toc}{chapter}{Glossar}
\renewcommand{\nomname}{Glossar}
\clearpage
\markboth{\nomname}{\nomname} %% see nomencl doc, page 9, section 4.1
\printnomenclature

%% index
\typeout{===== Section: index}
\printindex

\HAWasurency

\end{document}
