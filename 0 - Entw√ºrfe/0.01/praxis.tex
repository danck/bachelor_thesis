\chapter{Spark in der Praxis}
Im Folgenden wird Apache Spark im Rahmen zweier grundsätzlich verschiedener Anwendungsfälle betrachtet. \\

Beispiel 1: Eine typische Anwendung mit verteilten lokalem Storage (HDFS) und Spark als "`Client"' eines bestehenden Yarn Clustermanagers. \textcolor{gray}{--- Commodity Hardware (Rasperry Pi Cluster). ---}\\


Beispiel 2: Eine untypische Anwendung mit verteiltem entfernten Storage und dem Spark Standalone Clustermanager. \textcolor{gray}{--- HPC Hardware ("`Thunder"' des Hamburger KlimaCampus). ---}\\

\section{Echtzeitbewertung von Twitter-Accounts nach ihrer Relvanz für Spark}
\textcolor{gray}{--- Fusion von Tweets und Mailinglisten 
https://spark.apache.org/docs/1.3.0/mllib-feature-extraction.html
Implementation auf einem Raspberry Pi Cluster mit HDFS und Yarn Clustermanager ---}\\
\subsection{Beschreibung des Problems}
\textcolor{gray}{--- Es sollen die beiden Spark Mailingslisten (Developer, User) zur Identifikation relevanter und aktueller Themen genutzt werden. Mit den so bewerteten Begriffen können wiederum Tweets bewertet werden. Mit den Tweets können dann ganze Accounts nach ihrer Relevanz beurteilt werden. ---}\\
\textcolor{gray}{--- Zwei Datenquellen: Tweets (Nahe-Echtzeit), Entwickler-Emails (Sporadisch) ---}\\
\textcolor{gray}{--- Stichworte: HDFS, Yarn, Rasperri Pi Cluster, Machine Learning, Feature Extraction, Big Data Life Cycle ---}\\

\subsection{Hardwarekontext und Performance-Basisdaten}
\textcolor{gray}{--- hier kommen die eingesetzten systeme, und relevante laufzeitmessungen (netzwerk, storage, cpu) hin ---}\\

\begin{figure}[htbp]
  \includesvg[width=\paperwidth]{versuchsaufbau}
	\caption{svg image}
\end{figure}

\begin{table}[ht]
\caption{---DUMMY--- Netzwerkdurchsatz} % title of Table
\centering % used for centering table
\begin{tabular}{c c c c} % centered columns (4 columns)
\hline\hline %inserts double horizontal lines
Nachrichtengröße & Worker $\rightarrow$ Worker & Master $\rightarrow$ Worker & Worker $\rightarrow$ Master \\ [0.5ex] % inserts table
%heading
\hline % inserts single horizontal line
1kB & 50ms & 837ms & 970ms \\ % inserting body of the table
64kB & 47ms & 877ms & 230ms \\
1MB & 31ms & 25ms & 415ms \\
64MB & 35ms & 144ms & 2356ms \\ [1ex] 
\hline %inserts single line
\end{tabular}
\label{table:nonlin} % is used to refer this table in the text
\end{table}

\subsection{Architekturübersicht}

\begin{center}
\begin{tikzpicture}
\begin{umlpackage}{Apache Spark}
\begin{umlcomponent}{A}
\umlbasiccomponent{B}
\umlbasiccomponent[y=-2]{C}

\umlrequiredinterface[interface=C-interface]{C}
\umlprovidedinterface[interface=B-interface, with port, distance=3cm, padding=2.5cm]{B}
\end{umlcomponent}
\umlbasiccomponent[x=-10,y=1]{D}
\end{umlpackage}
\umlbasiccomponent[x=3,y=-7.5]{E}
\umlbasiccomponent[x=-2, y=-9]{F}
\umlbasiccomponent[x=-7,y=-8]{G}
\umlbasiccomponent[x=-7,y=-11]{H}

\umlassemblyconnector[interface=DA, with port, name=toto]{D}{A}
\umldelegateconnector{A-west-port}{B-west-interface}
\umlVHVassemblyconnector[interface=AE, with port]{A}{E}
\umlHVHassemblyconnector[interface=EF, with port, first arm]{E}{F}
\umlHVHassemblyconnector[interface=GHF, with port, arm2=-2cm, last arm]{G}{F}
\umlHVHassemblyconnector[with port, arm2=-2cm, last arm]{H}{F}

\umlnote[x=-4, y=4, width=3.4cm]{B-west-interface}{Hier ist B-west-interface}
\umlnote[x=2, y=4, width=3.4cm]{C-east-interface}{Hier ist C-east-interface}
\umlnote[x=-8.5, y=-2, width=3.4cm]{toto-interface}{Hier ist toto-interface}
\umlnote[x=-5.5, y=-4.5, width=3.4cm]{A-south-port}{Hier ist A-south-port}
\umlnote[x=-1, y=-6, width=3.4cm]{AE-interface}{Hier ist AE-interface}
\umlnote[x=2, y=-11, width=3.4cm]{F-east-port}{Hier ist F-east-port}
\end{tikzpicture}
\end{center}


\subsection{Detailierte Lösungsbeschreibung}
\textcolor{gray}{--- hier kommen diagramme und codeschnipsel hin ---}\\

\begin{lstlisting}[language=Scala, caption=Treiber für Testanwendung (Programmiersprache Scala)]
import org.apache.spark.SparkContext
import org.apache.spark.SparkContext._
import org.apache.spark.SparkConf

object ScalaApp {
  val my_spark_home = "/home/daniel/projects/spark-1.1.0"

  def main(args: Array[String]): Unit = {
    val logFile = my_spark_home + "/README.md"

    val conf = new SparkConf().setAppName("ScalaApp")
    val sc = new SparkContext(conf)

    val parList1 = sc.parallelize(List(1,2,3,4,5,6))
    val parList2 = sc.parallelize(List(5,6,7,8,9,10))
    val str1 = 
      "RDD1: \%s".format(parList1.collect().deep.mkString(" "))
    val str2 = 
      "RDD2: \%s".format(parList2.collect().deep.mkString(" "))
    val str3 = 
      "# of RDD1: \%s".format(parList1.count())
    val str4 = 
      "Intersect: \%s".format(parList1.intersection(parList2).collect()
    val str5 = 
      "Intersect: \%s".format(parList1.cartesian(parList2).collect()
  }
}
\end{lstlisting}

%%%%%%%%%%%%%%%%%%

\begin{center}
\begin{tikzpicture}
\begin{umlseqdiag}
\umlactor[class=A]{a}
\umldatabase[class=B, fill=blue!20]{b}
\umlmulti[class=C]{c}
\umlobject[class=D]{d}
\begin{umlcall}[op=opa(), type=synchron, return=0]{a}{b}
\begin{umlfragment}
\begin{umlcall}[op=opb(), type=synchron, return=1]{b}{c}
\begin{umlfragment}[type=alt, label=condition, inner xsep=8, fill=green!10]
\begin{umlcall}[op=opc(), type=asynchron, fill=red!10]{c}{d}
\end{umlcall}
\begin{umlcall}[type=return]{c}{b}
\end{umlcall}
\umlfpart[default]
\begin{umlcall}[op=opd(), type=synchron, return=3]{c}{d}
\end{umlcall}
\end{umlfragment}
\end{umlcall}
\end{umlfragment}
\begin{umlfragment}
\begin{umlcallself}[op=ope(), type=synchron, return=4]{b}
\begin{umlfragment}[type=assert]
\begin{umlcall}[op=opf(), type=synchron, return=5]{b}{c}
\end{umlcall}
\end{umlfragment}
\end{umlcallself}
\end{umlfragment}
\end{umlcall}
\umlcreatecall[class=E, x=8]{a}{e}
\begin{umlfragment}
\begin{umlcall}[op=opg(), name=test, type=synchron, return=6, dt=7, fill=red!10]{a}{e}
\umlcreatecall[class=F, stereo=boundary, x=12]{e}{f}
\end{umlcall}
\begin{umlcall}[op=oph(), type=synchron, return=7]{a}{e}
\end{umlcall}
\end{umlfragment}
\end{umlseqdiag}
\end{tikzpicture}
\end{center}


\subsection{Ergebnisse}
\textcolor{gray}{--- Tabellen und Diagramme Ergebnissen, evt. Skalierungsverhalten ---}
\textcolor{gray}{--- Bewertung ---}

\section{Evaluierung einer spark-basierten Implementation von CDOs auf einem HPC Cluster mit nicht-lokalem Storage}
\textcolor{gray}{--- Implementation ausgewählter CDOs (sehr wenige, möglicherweise nur 1-2) mit der Core-API von Spark. Testlauf auf einem HPC Cluster mit nicht-lokalem, allerdings per Infiniband angeschlossenen Storage.
Insbesondere Betrachtung des Skalierungsverhaltens und der "`Sinnhaftigkeit"'. ---}

\subsection{Beschreibung des Problems}
\textcolor{gray}{--- Erläuterung von CDOs (Climate Data Operators). ---}
\subsection{Hardwarekontext und Performance-Basisdaten}
\textcolor{gray}{--- hier kommen die eingesetzten systeme, und relevante laufzeitmessungen (netzwerk, storage, cpu) hin ---}
\subsection{Architekturübersicht}
\textcolor{gray}{--- hier kommen Verteilungs- und Komponentendiagramm hin ---}
\subsection{Detailierte Lösungsbeschreibung}
\textcolor{gray}{--- hier kommen laufzeitdiagramme und codeschnipsel hin ---}
\subsection{Ergebnisse}
\textcolor{gray}{--- Tabellen und Diagramme Ergebnissen, evt. Skalierungsverhalten ---}
\textcolor{gray}{--- Bewertung ---}
